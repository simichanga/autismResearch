\chapter{Introduction}

\section{Abstract}

A condition that needs a lot of care, autism is largely misunderstood, especially at a young age. 
Imagination of children with ASDs is captive with no way to be expressed, having difficulties in developing
an understanding of spoken language so they resort to nonverbal communication. 

This research paper aims to study patterns in cognitive behavior of people with ASD to see if we can facilitate 
them with better means of education through Game Theory. To support the research we implemented neural networks 
to see if/ how much people with ASDs can benefit from such an approach. By developing a program like this we want 
to also help medical professionals with a better understanding of them. Besides this, the program can help in 
improving their communication skills /and find a better way to communicate ideas and emotions.

{Key Words:}
ASD, children, communication, AI, neural networks, graph theory, game theory 

\section{Introduction}

A condition that needs a lot of care, autism is largely misunderstood, especially at a young age. 
It makes it difficult to communicate, thus dissociating people with ASD from the outside world.
Their imagination is captive with no way to be expressed, having difficulties in developing
an understanding of spoken language so they resort to nonverbal communication. This is often
misunderstood, slowly developing and accumulating frustration that can eventually be released
in an unhealthy manner, fact that leads to rigidity, anxiety and even depression when it comes to
social interactions. ASDs have a negative effect on children's developing education, their goals
and strategies to accomplish them.

In this field of study there are limited resources dedicated to autism, especially for children,
that's why we're aiming with this research paper to study patterns in cognitive behavior of people
with ASD to see if we can facilitate them with better means of education through Game Theory.
To support the research we implemented neural networks to see if/ how much people with ASDs can benefit
from such an approach.

By developing a program like this we want to also help medical professionals with a better understanding
of them. Besides this, the program can help in improving their communication skills and find a better
way to communicate ideas and emotions.